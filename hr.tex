%% LyX 2.2.3 created this file.  For more info, see http://www.lyx.org/.
%% Do not edit unless you really know what you are doing.
\documentclass[english]{article}
\usepackage[T1]{fontenc}
\usepackage[latin9]{inputenc}
\usepackage{geometry}
\geometry{verbose,tmargin=2cm,bmargin=2cm,lmargin=2cm,rmargin=2cm}
\setlength{\parskip}{6pt}
\setlength{\parindent}{0pt}
\usepackage{babel}
\usepackage{url}
\usepackage{amsmath}
\usepackage{amssymb}
\usepackage{graphicx}
\usepackage{xspace}
\usepackage[authoryear]{natbib}
\usepackage[unicode=true]
 {hyperref}


\begin{document}
David Bryant's example 1 (slightly different notation):
\begin{eqnarray}
  u & \sim & U[-a,a] \mbox{\hskip 5em rand. \#}   \\
    x_i^{\prime} & = & x_i e ^{u} \mbox{\hskip 6.5em determinstic} \\
    x_i & = & x_i^{\prime} e ^{u^{\prime}} \mbox{\hskip 6em eqn for reverse}  \\
    u^{\prime} & =&  \ln[x_i] - \ln[x_i^{\prime}]  \mbox{\hskip 3em rephrase for $u^{\prime}$ on lhs}\\
    & = & \ln[x_i] - \ln[x_i e ^{u}] \\
    & = & \ln[x_i] - \ln[x_i] - u \\
    & = & -u \\
  g(u) = g\left(u^{\prime}\right) & = & \frac{1}{2a} \\
  g\left(u^{\prime}\right) / g(u) & = & 1 \\
  \frac{\partial x_i^{\prime}}{\partial x_i} & = & e^u \\
  \frac{\partial x_i^{\prime}}{\partial x_j} & = & 0 \\
  \frac{\partial x_i^{\prime}}{\partial u} & = & x_i e ^{u}\\
  \frac{\partial u^{\prime}}{\partial x_{-}} & = & 0\\
  \frac{\partial u^{\prime}}{\partial u} & = & -1\\
  J & = & \left|\begin{array}{ccccc}
    e^u & 0 & \ldots & 0 & x_1 e ^{u} \\
    0 & e^u  & \ldots & 0 & x_2 e ^{u} \\
    \vdots & \vdots & \ddots &  \vdots & \vdots \\
    0 & 0 & \ldots & e^u  & x_n e ^{u} \\
    0 & 0 & \ldots & 0 & -1
  \end{array}\right|\\
  \det(J) & = & -e^{nu} \\
  |\det(J)| & = & e^{nu}
\end{eqnarray}
If you say that $m = e^u$, then the HR is $m^n$

We could think of the same move in terms of drawing the multiplier:
\begin{eqnarray}
  u & \sim & U[-a,a] \mbox{\hskip 5em rand. \#}   \\
  m & = & e ^{u}\\
\end{eqnarray}
and then:
\begin{eqnarray}
    x_i^{\prime} & = & x_i m \mbox{\hskip 6.5em determinstic} \\
    x_i & = & x_i^{\prime} m^{\prime} \mbox{\hskip 6em eqn for reverse}  \\
    m^{\prime} & =&  \frac{x_i}{x_i^{\prime}}  \mbox{\hskip 3em rephrase for $u^{\prime}$ on lhs}\\
    & = & m^{-1} \\
  \frac{\partial x_i^{\prime}}{\partial x_i} & = & m \\
  \frac{\partial x_i^{\prime}}{\partial x_j} & = & 0 \\
  \frac{\partial x_i^{\prime}}{\partial m} & = & x_i m\\
  \frac{\partial m^{\prime}}{\partial x_{-}} & = & 0\\
  \frac{\partial m^{\prime}}{\partial m} & = & -m^{-2}\\
  J & = & \left|\begin{array}{ccccc}
    m & 0 & \ldots & 0 & x_1 m \\
    0 & m  & \ldots & 0 & x_2 m \\
    \vdots & \vdots & \ddots &  \vdots & \vdots \\
    0 & 0 & \ldots & m  & x_n m \\
    0 & 0 & \ldots & 0 & -m^{-2}
  \end{array}\right|\\
  \det(J) & = & -m^{n-2} \\
  |\det(J)| & = & m^{n-2}
\end{eqnarray}
So, it looks like we are headed to a different Hastings ratio for the same move (this time
with a Jacobian of $m^{n-2}$ instead of $m^{n}$, but$\ldots$ 

If we calculate the determinant in terms of the $m$ parameterization, then we need to have the 
    probability density $g()$ in terms of $m$, too.
$u$ is drawn from a uniform, but $m$ is not.

We can see that (for the range $e^{-a}\leq m \leq e^a$), the CDF is:
\begin{eqnarray}
 F(m) & = & \frac{\ln[m] - \ln[e^{-a}]}{2a} \\
 & = & \frac{\ln[m] + a}{2a}
\end{eqnarray}
by matching the probaility of $u$ (in the range $-a\leq u \leq u$) that is used to generate $m$""
\begin{eqnarray}
 F(u) & = & \frac{u + a}{2a}
\end{eqnarray}

To find the pdf of $m$ we differentiate $F(m)$ wrt $m$:
\begin{eqnarray}
 f(m) & = & \frac{d F(m)}{d m} \\
   \frac{d F(m)}{d m} & =&  \frac{d (\ln[m] + a)/sa)}{dm} \\
    &  = & \frac{d (\ln[m])/sa)}{dm} \\
    & = & \frac{1}{2am} 
\end{eqnarray}

Thus
\begin{eqnarray}
  g\left(m^{\prime}\right) / g(m) & = & \left( \frac{1}{2am^{\prime}} \right) / \left( \frac{1}{2am} \right)\\
  & = & \frac{2am}{2am^{\prime}} \\
  & = & \frac{m}{m^{\prime}} \\
  & = & m^2
\end{eqnarray}
Thus the Hastings ratio is: $m^2 m^{n-2} = m^n$, once again.

So it doesn't matter if we think of this as drawing $u$ or drawing $m$, but we do have to be internally consistent.

If we change the move to draw $m\sim U[e^{-a}, e^a]$ then the probability densities $g$ change
and the HR becomes $m^{n-2}$.

So, if we draw the multiplier as $m = e^{\lambda(2u - 1)}$ (i.e., $u \sim U[-1,
1]$ and $m = e^{\lambda u}$), the Hasting ratio is $m^n$.

'RateMixer' draws $m \sim U[\lambda, 1/\lambda]$, which would make the
Hastings ratio $m^{n-2}$, which is what it returns.

\end{document}
