%% LyX 2.2.3 created this file.  For more info, see http://www.lyx.org/.
%% Do not edit unless you really know what you are doing.
\documentclass[english]{article}
\usepackage[T1]{fontenc}
\usepackage[latin9]{inputenc}
\usepackage{geometry}
\geometry{verbose,tmargin=2cm,bmargin=2cm,lmargin=2cm,rmargin=2cm}
\setlength{\parskip}{6pt}
\setlength{\parindent}{0pt}
\usepackage{babel}
\usepackage{url}
\usepackage{amsmath}
\usepackage{amssymb}
\usepackage{graphicx}
\usepackage{xspace}
\usepackage[authoryear]{natbib}
\usepackage[unicode=true]
 {hyperref}


\begin{document}
David Bryant's example 1 (slightly different notation):
\begin{eqnarray}
  u & \sim & U[-a,a] \mbox{\hskip 5em rand. \#}   \\
    x_i^{\prime} & = & x_i e ^{u} \mbox{\hskip 6.5em determinstic} \\
    x_i & = & x_i^{\prime} e ^{u^{\prime}} \mbox{\hskip 6em eqn for reverse}  \\
    u^{\prime} & =&  \ln[x_i] - \ln[x_i^{\prime}]  \mbox{\hskip 3em rephrase for $u^{\prime}$ on lhs}\\
    & = & \ln[x_i] - \ln[x_i e ^{u}] \\
    & = & \ln[x_i] - \ln[x_i] - u \\
    & = & -u \\
  g(u) = g\left(u^{\prime}\right) & = & \frac{1}{2a} \\
  g\left(u^{\prime}\right) / g(u) & = & 1 \\
  \frac{\partial x_i^{\prime}}{\partial x_i} & = & e^u \\
  \frac{\partial x_i^{\prime}}{\partial x_j} & = & 0 \\
  \frac{\partial x_i^{\prime}}{\partial u} & = & x_i e ^{u}\\
  \frac{\partial u^{\prime}}{\partial x_{-}} & = & 0\\
  \frac{\partial u^{\prime}}{\partial u} & = & -1\\
  J & = & \left|\begin{array}{ccccc}
    e^u & 0 & \ldots & 0 & x_1 e ^{u} \\
    0 & e^u  & \ldots & 0 & x_2 e ^{u} \\
    \vdots & \vdots & \ddots &  \vdots & \vdots \\
    0 & 0 & \ldots & e^u  & x_n e ^{u} \\
    0 & 0 & \ldots & 0 & -1
  \end{array}\right|\\
  \det(J) & = & -e^{nu} \\
  |\det(J)| & = & e^{nu}
\end{eqnarray}
If you say that $m = e^u$, then the HR is $m^n$
\end{document}
